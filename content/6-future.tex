%----------------------------------------------------------------------------------------
% Future
%----------------------------------------------------------------------------------------

\section{Future Considerations}

\subsection{Quarterback Controller}

\quad While we faced some trouble moving onto the next iteration of our model, in our future work we plan to add more complexity to the Quarterback’s controller. This includes finishing the implementation of forward and backwards movement, so that the Quarterback can move more freely. This includes a loop which starts with a control that switches direction based on conditions that include how far the Wide Receiver is and how far the Defensive Line is. \\ 

Another continuation of this model would be to define a maximum passing range for the Quarterback, because currently, the Quarterback passes as soon as the Wide Receiver is open. The average Quarterback can only throw between 60 to 80 yards (or 180 to 240 feet), so it is not entirely realistic if the Quarterback is at yard 0 and passed to the Wide Receiver at yard 300 (which our model currently allows). Our constraint on the Quarterback and Wide Receiver would be that they are within 200 feet of each other. To calculate the distance, we would simply subtract their y-coordinates. However, when we move into two dimensions, as discussed in the next section this distance would be the Euclidean distance between them. \\

% ༼ つ ◕_◕ ༽つ EUCLIDEAN ~~~

We have to be careful with how we define our "maximum passing range". There are two different notions of maximum that we can consider here: a restriction on physics or a heuristic. The restriction on physics allows the Quarterback to travel backwards while he is within 200 feet of the Wide Receiver. While this will allow the Quarterback to avoid getting tackled by the Defensive Line for as long as possible, it may make it harder for the offense to win in the case that he travels too far back and can never re-enter a 200 foot range of the Wide Receiver. After all, if the Quarterback leaves the range of which he is physically capable of passing, it takes time and effort to get back to that range. On the other hand, the heuristic idea would define the "maximum" to be half of the quarterback's physical capability (in this case, 100 feet). The motivation for this is so that the Quarterback never has to worry about re-entering passing range like before. With this heuristic (after some further formalization), we can say that the Quarterback will always be within passing range. However, this constraint also makes it harder for the Quarterback to avoid getting tackled by the Defensive Line, since he is not traveling as far back. Both notions have their pros and cons, so we would test both out. \\

Below, we have the post-collision phase of what our model would have looked like, had we implemented the ideas we just discussed here. Here, we use a loop to emulate a Quarterback constantly checking whether or not he is able to pass. We want our model to stop evolving once the Quarterback passes, so we use \textbf{3.1.4 The Equivalence of Openness and Passing} to stop the ODE once \texttt{isOpen(yWR,yLB,buffer) <-> false}. Furthermore, since we are using a diamond modality, as discussed in \textbf{5.1 Proving Diamond Modalities}, we use \texttt{yQB < yD} to ensure our safety condition is true throughout the duration of the ODE.

\begin{lstlisting}
    {
      /* If within maxPass, move backwards. Else, move forward */
      ?(dist(yQB, yWR) <= maxPass); dyQB := -1; ++ dyQB := 1;
      /* Keep evolving; stop once WR open or realism breaks */
      { yQB' = dyQB*vyQB,
        yD'  = (dyA*vyA + dyD*vyD)/2, /* Dampened Movement */
        yWR' = dyWR*vyWR, yLB' = dyLB*vyLB,
        t' = 1
      & yA >= yD /* Collided */
      & !isOpen(yWR, yLB, buffer) /* WR not open */
      & t <= 1 /* Time Triggered */
      & yQB < yD /* Model responsible for domain constraint */
      }
    }*
\end{lstlisting}

\subsection{Two Dimensions}

\quad Currently, our model is simplified down to one dimension (as discussed in \textbf{3.1.3 One Dimension}). However, it is indeed more realistic to allow players, like the Wide Receiver, Quarterback, and Linebacker to travel in the x-direction. This is reflected in real football when players move along sharp diagonals and turns in attempts to throw off their opponents. \\

Therefore, in the future, we would start by introducing the x-direction for the Wide Receiver to enable movement around the Linebacker. Then, we would introduce the x-direction to the Linebacker’s controllers to model defending the Wide Receiver more accurately. The final step would be integrating a controller in the x-direction for the Quarterback to enable getting closer to the Wide Receiver in the x-direction, so as to remain within passing range.

\subsection{Different Defensive Line}

\quad Currently, our Offensive and Defensive Lines' actions are modelled as a dampened line moving slowly towards the Quarterback. If we want to make this even more realistic, we would want to model the scenario that an additional Linebacker or one of the linemen are able to get past the Offensive Line. In this scenario, we would see a sack happen, where the Quarterback gets tackled before he gets a chance to pass it, or behind the line of scrimmage. One way we could model this is, assuming we modelled multiple linemen individually instead of as a "collective", is choose with some random probability one player to "break past". \\

Going off the idea of multiple linemen, another thing we would want to consider is modelling a curve. Generally you'll see that linemen are not an exact straight line, but rather a curve: the outside more curved than the inside. This is because they all have the common goal of tackling the Quarterback and likely will take the shortest distance to accomplish that, instead of moving directly forward. However, since our model is one dimensional, a straight line sufficed for our problem.